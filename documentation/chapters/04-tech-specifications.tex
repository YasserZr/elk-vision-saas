\chapter{Technical Specifications}

\section{Technology Stack Overview}

This section details all technologies used in ELK Vision SaaS with version numbers and justifications for each choice.

\section{Backend Technologies}

\begin{table}[H]
    \centering
    \begin{tabular}{|l|l|l|}
        \hline
        \textbf{Technology} & \textbf{Version} & \textbf{Purpose} \\
        \hline
        Python & 3.11+ & Core backend language \\
        Django & 4.2 LTS & Web framework \\
        Django REST Framework & 3.14 & RESTful API \\
        Django Channels & 4.0 & WebSocket support \\
        Daphne & 4.0 & ASGI server \\
        Celery & 5.3 & Async task processing \\
        PyMongo & 4.5 & MongoDB driver \\
        Psycopg2 & 2.9 & PostgreSQL adapter \\
        elasticsearch-dsl & 8.11 & Elasticsearch client \\
        redis & 5.0 & Redis client \\
        \hline
    \end{tabular}
    \caption{Backend Technology Stack}
    \label{tab:backend-tech}
\end{table}

\subsection{Django Framework Justification}

Django was chosen for the backend because:
\begin{itemize}
    \item \textbf{Batteries Included}: Built-in authentication, admin, ORM, and security features
    \item \textbf{Django REST Framework}: Industry-standard API development toolkit
    \item \textbf{Django Channels}: Native WebSocket support with channel layers
    \item \textbf{Ecosystem}: Rich library support for Elasticsearch, Celery, and monitoring
    \item \textbf{LTS Support}: Django 4.2 has long-term support until April 2026
\end{itemize}

\section{Frontend Technologies}

\begin{table}[H]
    \centering
    \begin{tabular}{|l|l|l|}
        \hline
        \textbf{Technology} & \textbf{Version} & \textbf{Purpose} \\
        \hline
        Node.js & 20 LTS & JavaScript runtime \\
        Next.js & 14.0 & React framework \\
        React & 18 & UI library \\
        TypeScript & 5 & Type-safe JavaScript \\
        Tailwind CSS & 3 & Utility-first CSS \\
        Recharts & 2.8 & Data visualization \\
        \hline
    \end{tabular}
    \caption{Frontend Technology Stack}
    \label{tab:frontend-tech}
\end{table}

\subsection{Next.js Framework Justification}

Next.js 14 was selected for the frontend because:
\begin{itemize}
    \item \textbf{App Router}: Modern file-based routing with React Server Components
    \item \textbf{Performance}: Automatic code splitting and optimized builds
    \item \textbf{Full-Stack}: API routes for health checks and server-side logic
    \item \textbf{TypeScript}: First-class TypeScript support for type safety
    \item \textbf{Developer Experience}: Fast refresh and excellent debugging tools
\end{itemize}

\section{Database Technologies}

\begin{table}[H]
    \centering
    \begin{tabular}{|l|l|l|l|}
        \hline
        \textbf{Database} & \textbf{Version} & \textbf{Purpose} & \textbf{Data Types} \\
        \hline
        Elasticsearch & 8.11 & Log search/storage & Indexed documents \\
        PostgreSQL & 15 & Application data & Users, sessions, config \\
        MongoDB & 7.0 & Document storage & Log metadata \\
        Redis & 7 & Cache/messaging & Sessions, Pub/Sub \\
        \hline
    \end{tabular}
    \caption{Database Technology Stack}
    \label{tab:database-tech}
\end{table}

\subsection{Multi-Database Justification}

Multiple databases are used to optimize for different workloads:

\begin{itemize}
    \item \textbf{Elasticsearch}: Optimized for full-text search and log analytics at scale
    \item \textbf{PostgreSQL}: ACID-compliant storage for critical application data
    \item \textbf{MongoDB}: Flexible document storage for variable log metadata
    \item \textbf{Redis}: In-memory operations for caching and real-time messaging
\end{itemize}

\section{ELK Stack}

\begin{table}[H]
    \centering
    \begin{tabular}{|l|l|l|}
        \hline
        \textbf{Component} & \textbf{Version} & \textbf{Purpose} \\
        \hline
        Elasticsearch & 8.11.1 & Log indexing and search \\
        Logstash & 8.11.1 & Log ingestion and transformation \\
        Kibana & 8.11.1 & Advanced visualization \\
        \hline
    \end{tabular}
    \caption{ELK Stack Versions}
    \label{tab:elk-tech}
\end{table}

\section{Infrastructure Technologies}

\begin{table}[H]
    \centering
    \begin{tabular}{|l|l|l|}
        \hline
        \textbf{Technology} & \textbf{Version} & \textbf{Purpose} \\
        \hline
        Docker & 24+ & Containerization \\
        Docker Compose & 2.20+ & Multi-container orchestration \\
        Nginx & 1.25 & Reverse proxy \\
        \hline
    \end{tabular}
    \caption{Infrastructure Technology Stack}
    \label{tab:infra-tech}
\end{table}

\section{Monitoring Stack}

\begin{table}[H]
    \centering
    \begin{tabular}{|l|l|l|}
        \hline
        \textbf{Technology} & \textbf{Version} & \textbf{Purpose} \\
        \hline
        Prometheus & 2.x & Metrics collection \\
        Grafana & 10.x & Metrics visualization \\
        Alertmanager & 0.26 & Alert management \\
        Flower & 2.x & Celery monitoring \\
        \hline
    \end{tabular}
    \caption{Monitoring Technology Stack}
    \label{tab:monitoring-tech}
\end{table}

\section{System Requirements}

\subsection{Minimum Requirements}

\begin{itemize}
    \item \textbf{CPU}: 4 cores
    \item \textbf{RAM}: 8 GB
    \item \textbf{Disk}: 50 GB free space
    \item \textbf{OS}: Linux, macOS, or Windows 10/11 with WSL2
\end{itemize}

\subsection{Recommended Requirements}

\begin{itemize}
    \item \textbf{CPU}: 8+ cores
    \item \textbf{RAM}: 16 GB
    \item \textbf{Disk}: 100 GB SSD
    \item \textbf{OS}: Ubuntu 22.04 LTS or newer
\end{itemize}

\section{External Dependencies}

\begin{itemize}
    \item \textbf{Docker Hub}: Container registry for official images
    \item \textbf{Elastic Registry}: ELK Stack container images
    \item \textbf{npm Registry}: Node.js packages
    \item \textbf{PyPI}: Python packages
\end{itemize}
