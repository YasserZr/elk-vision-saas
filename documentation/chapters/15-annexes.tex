\chapter{Annexes}

\section{API Reference}

\subsection{Authentication Endpoints}

\begin{table}[H]
    \centering
    \begin{tabular}{|l|l|l|}
        \hline
        \textbf{Method} & \textbf{Endpoint} & \textbf{Description} \\
        \hline
        POST & /api/auth/register/ & Create user account \\
        POST & /api/auth/login/ & Obtain auth token \\
        POST & /api/auth/logout/ & Invalidate token \\
        GET & /api/auth/user/ & Get current user \\
        \hline
    \end{tabular}
    \caption{Authentication API Endpoints}
    \label{tab:auth-api}
\end{table}

\subsection{Logs Endpoints}

\begin{table}[H]
    \centering
    \begin{tabular}{|l|l|l|}
        \hline
        \textbf{Method} & \textbf{Endpoint} & \textbf{Description} \\
        \hline
        GET & /api/logs/ & List logs (paginated) \\
        POST & /api/logs/ & Create log entry \\
        GET & /api/logs/\{id\}/ & Get log details \\
        POST & /api/logs/upload/ & Upload log file \\
        POST & /api/logs/bulk/ & Create multiple logs \\
        GET & /api/logs/search/ & Search with filters \\
        GET & /api/logs/stats/ & Get statistics \\
        \hline
    \end{tabular}
    \caption{Logs API Endpoints}
    \label{tab:logs-api}
\end{table}

\subsection{WebSocket Endpoints}

\begin{table}[H]
    \centering
    \begin{tabular}{|l|l|}
        \hline
        \textbf{Endpoint} & \textbf{Description} \\
        \hline
        ws://host/ws/logs/stream/ & Real-time log streaming \\
        ws://host/ws/notifications/ & User notifications \\
        ws://host/ws/metrics/ & Live metrics updates \\
        \hline
    \end{tabular}
    \caption{WebSocket Endpoints}
    \label{tab:ws-api}
\end{table}

\section{Environment Variables Reference}

\begin{longtable}{|l|l|l|}
    \hline
    \textbf{Variable} & \textbf{Default} & \textbf{Description} \\
    \hline
    SECRET\_KEY & - & Django secret key \\
    DEBUG & False & Debug mode \\
    ALLOWED\_HOSTS & localhost & Allowed hosts \\
    \hline
    POSTGRES\_DB & elk\_vision & Database name \\
    POSTGRES\_USER & postgres & Database user \\
    POSTGRES\_PASSWORD & - & Database password \\
    \hline
    MONGO\_HOST & mongodb & MongoDB host \\
    MONGO\_USER & admin & MongoDB user \\
    MONGO\_PASSWORD & - & MongoDB password \\
    MONGO\_DB\_NAME & elk\_vision & MongoDB database \\
    \hline
    REDIS\_HOST & redis & Redis host \\
    REDIS\_PORT & 6379 & Redis port \\
    REDIS\_PASSWORD & - & Redis password \\
    \hline
    ELASTICSEARCH\_HOST & localhost:9200 & ES host \\
    ELASTICSEARCH\_PASSWORD & - & ES password \\
    \hline
    NEXT\_PUBLIC\_API\_URL & http://localhost:8000 & Backend URL \\
    NEXT\_PUBLIC\_WS\_URL & ws://localhost:8000 & WebSocket URL \\
    \hline
    \caption{Environment Variables}
    \label{tab:env-vars}
\end{longtable}

\section{Port Reference}

\begin{table}[H]
    \centering
    \begin{tabular}{|l|l|l|}
        \hline
        \textbf{Port} & \textbf{Service} & \textbf{Protocol} \\
        \hline
        3000 & Next.js Frontend & HTTP \\
        8000 & Django Backend & HTTP/WebSocket \\
        5432 & PostgreSQL & TCP \\
        27017 & MongoDB & TCP \\
        6379 & Redis & TCP \\
        9200 & Elasticsearch & HTTP \\
        5000 & Logstash & TCP/UDP \\
        5044 & Logstash Beats & TCP \\
        5601 & Kibana & HTTP \\
        9090 & Prometheus & HTTP \\
        3001 & Grafana & HTTP \\
        5555 & Flower & HTTP \\
        80/443 & Nginx & HTTP/HTTPS \\
        \hline
    \end{tabular}
    \caption{Service Port Mapping}
    \label{tab:ports}
\end{table}

\section{Docker Commands Reference}

\begin{lstlisting}[language=bash, caption=Common Docker Commands]
# Start all services
docker compose up -d

# Stop all services
docker compose down

# View logs
docker compose logs -f [service]

# Execute command in container
docker compose exec backend python manage.py shell

# Rebuild services
docker compose build [service]

# Check service health
docker compose ps

# Scale services
docker compose up -d --scale backend=3
\end{lstlisting}

\section{References}

\subsection{Official Documentation}

\begin{itemize}
    \item Django Documentation: \url{https://docs.djangoproject.com/}
    \item Django REST Framework: \url{https://www.django-rest-framework.org/}
    \item Django Channels: \url{https://channels.readthedocs.io/}
    \item Next.js Documentation: \url{https://nextjs.org/docs}
    \item Elasticsearch Guide: \url{https://www.elastic.co/guide/}
    \item Logstash Reference: \url{https://www.elastic.co/guide/en/logstash/current/}
    \item Kibana Guide: \url{https://www.elastic.co/guide/en/kibana/current/}
    \item Docker Documentation: \url{https://docs.docker.com/}
\end{itemize}

\subsection{Project Resources}

\begin{itemize}
    \item GitHub Repository: \url{https://github.com/YasserZr/elk-vision-saas}
    \item API Documentation: \code{http://localhost:8000/api/docs/}
    \item OpenAPI Schema: \code{http://localhost:8000/api/schema/}
\end{itemize}

\section{Glossary}

\begin{description}
    \item[ASGI] Asynchronous Server Gateway Interface - Python async web server protocol
    \item[Celery] Distributed task queue for Python
    \item[ELK Stack] Elasticsearch, Logstash, Kibana - log management stack
    \item[Pub/Sub] Publish/Subscribe messaging pattern
    \item[REST] Representational State Transfer - API architectural style
    \item[WebSocket] Full-duplex communication protocol
    \item[WSGI] Web Server Gateway Interface - Python web server protocol
\end{description}

\section{License}

This project is licensed under the MIT License. See the LICENSE file in the repository for details.

\vspace{1cm}
\begin{center}
    \textbf{Document Generated: \today}
\end{center}
